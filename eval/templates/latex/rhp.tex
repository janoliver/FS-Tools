%  Liste der Dateien:
%  
%  # rhp.sty: Stile, befehle, eingebundene Pakete usw. der RHP. Sozusagen die gesamte Präambel
%  # vorwort.eps: Das Bildchen, was diese Statistikkästen erläutert.
%  # titel.tex: Titelseite
%  
%  TODOs:
%  
%  # rhp.tex sollte so einfach wie möglich gehalten werden. Daher: möglichst viele Makros
%    schreiben, um die Struktur der RHP zu vereinfachen. Bsp: DIe ganzen newpages usw. 
%  # vorwort.eps sollte eine tikz oder pstricks grafik werden. 
%  # ALLES UTF-8 bitte!
%  # \Stats überarbeiten vielleicht...



\documentclass[oneside,12pt, headinclude=true, footinclude=false]{scrreprt}

% Alle Stildefinitionen, Pakete und Befehle werden in rhp.sty definiert.
\usepackage{rhp}

\begin{document}

%\pagestyle{empty}

\BLOCK{for vl in vorlesungen}

\BLOCK{if vl.antwortboegen.all()|count > 0}

\setcounter{page}{1}
\Kopfklein{\VAR{vl.name}}{\VAR{ vl.dozenten.all()|concat}}

\genehm{\VAR{ vl.dozenten.all()|concat}}
\newpage

\centerline{\bf Abgegebene Fragebögen: \VAR{vl.antwortboegen.all()|count}}

\begin{center}
  \begin{tabular}{p{3cm}p{3cm}p{5cm}l}
    {\bf Semesterverteilung:}&&{\bf Studiengänge:}& \\
    \BLOCK{for index in range(vl.studiengang_dist()|max(vl.semester_dist()))}
    \BLOCK{if vl.semester_dist()|count > index -}
      Semester \VAR{vl.semester_dist()[index]['key']}: & \VAR{vl.semester_dist()[index]['val']} 
    \BLOCK{- else} & \BLOCK{endif -}
    &
    \BLOCK{- if vl.studiengang_dist()|count > index -}
      \VAR{vl.studiengang_dist()[index]['key']}: & \VAR{vl.studiengang_dist()[index]['val']} 
    \BLOCK{- else} & \BLOCK{endif} \\
    \BLOCK{endfor}
  \end{tabular}
\end{center}
\Strich

% auswertung
\BLOCK{for fragenset in vl.fragebogen.fragensets.all()}
  \BLOCK{for frage in fragenset.fragen(fragen)} 
    \BLOCK{if not frage.fragentyp.texttype and frage.total(vl) > 0}
      \Frage{\VAR{frage.text}}{\VAR{frage.total(vl)}}
      \BLOCK{- for option in frage.fragentyp.fast_optionen(optionen) -}
        \BLOCK{- if loop.first -}
          {\VAR{option.text|latexify}}{\Stats 
        \BLOCK{- endif -}
        {\VAR{(option.value(vl, frage) * 100)|int}}
        \BLOCK{- if loop.last -}
          {\VAR{((frage.mean(vlu)-1) * 100)|int}}
          {\VAR{((frage.mean(vl)-1) * 100)|int}}}
          {\VAR{option.text|latexify}}
        \BLOCK{- endif -}
      \BLOCK{- endfor} \\
    \BLOCK{- endif}
  \BLOCK{endfor}
\BLOCK{endfor}

\Strich
\centerline{\large Liste aller Kommentare:}

\BLOCK{for fragenset in vl.fragebogen.fragensets.all() -}
  \BLOCK{- for frage in fragenset.fragen(fragen) -}
    \BLOCK{- if frage.fragentyp.texttype and frage.id != 34}
      {\bf \VAR{frage.text}}
      \BLOCK{if frage.antworten(vl)|count > 0 -}
        \begin{itemize}
          \setlength{\itemsep}{0cm} \setlength{\parsep}{0cm}
        \BLOCK{for k in frage.antworten(vl)}
          \item \VAR{k.text|latexify}
        \BLOCK{endfor}
        \end{itemize}
      \BLOCK{else}
        \begin{itemize}\item Keine Kommentare\end{itemize}
      \BLOCK{endif -}
    \BLOCK{- endif -}

    \BLOCK{- if frage.fragentyp.texttype and frage.id == 34}
      {\bf \VAR{frage.text}}
      \BLOCK{if vl.tutoren.all()|count > 0}
        \begin{itemize}
        \BLOCK{for tutor in vl.tutoren.all()}
          \item Gruppe von \VAR{tutor.name}
          \BLOCK{if frage.antworten(vl, tutor)|count > 0 -}
            \begin{itemize}
              \setlength{\itemsep}{0cm} \setlength{\parsep}{0cm}
            \BLOCK{for k in frage.antworten(vl, tutor)}
              \item \VAR{k.text|latexify}
            \BLOCK{endfor}
            \end{itemize}
          \BLOCK{- else -}
            \begin{itemize}\item Keine Kommentare\end{itemize}
          \BLOCK{endif -}
        \BLOCK{- endfor -}
        \end{itemize}
      \BLOCK{else}
        \begin{itemize}\item Keine Kommentare\end{itemize}
      \BLOCK{- endif}
    \BLOCK{endif}
  \BLOCK{- endfor -}
\BLOCK{- endfor}

\BLOCK{for fragenset in vl.fragebogen.fragensets.all() -}
  \BLOCK{- for frage in fragenset.fragen(fragen) -}
    \BLOCK{- if frage.fragentyp.texttype and frage.id == 34}
      \BLOCK{if vl.tutoren.all()|count > 0}
        \BLOCK{for tutor in vl.tutoren.all()}
          \BLOCK{if frage.antworten(vl, tutor)|count > 0 -}
            \newpage
            \genehm{\VAR{tutor.name}}
            \newpage
            
            \centerline{\bf Abgegebene Fragebögen: \VAR{vl.antwortboegen.all()|count}}
            
            \begin{center}
              \begin{tabular}{p{3cm}p{3cm}p{5cm}l}
                {\bf Semesterverteilung:}&&{\bf Studiengänge:}& \\
                \BLOCK{for index in range(vl.studiengang_dist()|max(vl.semester_dist()))}
                \BLOCK{if vl.semester_dist()|count > index -}
                Semester \VAR{vl.semester_dist()[index]['key']}: & \VAR{vl.semester_dist()[index]['val']} 
                \BLOCK{- else} & \BLOCK{endif -}
                &
                \BLOCK{- if vl.studiengang_dist()|count > index -}
                \VAR{vl.studiengang_dist()[index]['key']}: & \VAR{vl.studiengang_dist()[index]['val']} 
                \BLOCK{- else} & \BLOCK{endif} \\
                \BLOCK{endfor}
              \end{tabular}
            \end{center}
            \Strich

            {\bf \VAR{frage.text}}\\
            Tutor: \VAR{tutor.name}
      
            \begin{itemize}
            \BLOCK{for k in frage.antworten(vl, tutor)}
              \item \VAR{k.text|latexify}
            \BLOCK{endfor}
            \end{itemize}
          \BLOCK{endif}
        \BLOCK{- endfor -}
      \BLOCK{endif}
    \BLOCK{endif}
  \BLOCK{- endfor -}
\BLOCK{- endfor}



\BLOCK{endif} % anzahl vorlesungen

\newpage

\BLOCK{endfor}

\end{document}

