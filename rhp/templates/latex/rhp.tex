%  Liste der Dateien:
%  
%  # rhp.sty: Stile, befehle, eingebundene Pakete usw. der RHP. Sozusagen die gesamte Präambel
%  # vorwort.eps: Das Bildchen, was diese Statistikkästen erläutert.
%  # titel.tex: Titelseite
%  
%  TODOs:
%  
%  # rhp.tex sollte so einfach wie möglich gehalten werden. Daher: möglichst viele Makros
%    schreiben, um die Struktur der RHP zu vereinfachen. Bsp: DIe ganzen newpages usw. 
%  # vorwort.eps sollte eine tikz oder pstricks grafik werden. 
%  # ALLES UTF-8 bitte!
%  # \Stats überarbeiten vielleicht...



\documentclass[twoside,12pt, headinclude=true, footinclude=false]{scrreprt}

% Alle Stildefinitionen, Pakete und Befehle werden in rhp.sty definiert.
\usepackage{rhp}



\begin{document}

%%%%% TITEL
\begin{titlepage} 

\begin{flushleft}
\newfont{\phifont}{wncyr10 at 350pt}
\mbox{  }
\vspace{3cm}

\centerline{\Huge \bf Renthofpostille}
\vspace{1.5cm}
\centerline{\large am FB Physik der}
\centerline{\large Philipps-Universität Marburg}
\vspace{3cm}
\centerline{
{\mbox{\phifont \hspace*{-2,7cm}F }
\hspace{-12,2cm}\raisebox{4cm}{\large \sf Fachschaft \hspace{1,4cm} Physik}}}
\vfill
Marburg, den \today
\end{flushleft} 

\end{titlepage}

\newpage
\thispagestyle{empty}
\newpage


%%%%% INHALTSVERZEICHNIS
\markboth{Renthofpostille}{Renthofpostille}
\clearpage
\tableofcontents
\clearpage

%%%%% Vorwort
\artikel{Vorwort}

Hallo liebe Leserin,

\autor{Deine Fachschaft}
\newpage

%%%%% ARTIKEL
\BLOCK{for a in artikel}
  \artikel{\VAR{a.titel}}
  \VAR{a.text}
  \autor{\VAR{a.autor}}
\BLOCK{endfor}


%%%%% VLU
\zuminhalt{Vorlesungsevaluation}

\markboth{Vorlesungsevaluation \VAR{vlu.name}}{Vorwort}
\centerline{\bf \huge Vorwort zur Vorlesungsumfrage}

\vspace{5mm}

%@TODO
Im Zeitraum \VAR{vlu.vlu_start|date} bis \VAR{vlu.vlu_end|date} wurde im Fachbereich  Physik  wieder eine
Vorlesungsumfrage durchgeführt. Befragt wurden Studierende aus allen
größeren Vorlesungen sowie einigen Spezialvorlesungen.

\vspace{5mm}

Die evaluierten Veranstaltungen und die Anzahl der abgegebenen Fragebögen:

% TODO: maschinell generieren
\begin{tabular}{ll@{ : }l}
  \BLOCK{for vl in vorlesungen}                              % loop über vorlesungen der VLU
  \VAR{vl.name} & \VAR{ vl.dozenten.all()|concat} & \VAR{ vl.antwortboegen.all()|count } \\
  \BLOCK{endfor}
\end{tabular}

\vspace{5mm}
\textbf{\large Zur Vorgehensweise bei der Umfrage:}
\vspace{0.05cm}

Die Fragebögen wurden in den Vorlesungen ausgeteilt, ausgefüllt und  direkt im Anschluss daran wieder eingesammelt.
Dadurch konnte eine Rücklaufquote von nahezu 100 Prozent erzielt werden. Für das Austeilen und Einsammeln, sowie
das Auswerten der Bögen sei an dieser Stelle noch einmal allen Beteiligten gedankt!

\vspace{5mm}
\textbf{\large Zu den Diagrammen:}
\vspace{0.05cm}

Die Kästchen hinter den Fragen entsprechen den Feldern auf den Fragebögen. Die Balkendiagramme geben die Häufigkeit
wieder, mit der das entsprechende Feld angekreuzt wurde. Die Höhe der Balken ist dabei auf die Gesamtzahl von gültigen
Stimmen normiert. Hinter dem Fragetext ist zusätzlich jeweils die Anzahl der bei dieser Fage gültigen Bögen vermerkt.

Die Mittelwertberechnung zu jeder Frage und Vorlesung ergibt sich aus den jeweils gültigen Stimmen und ist als schwarzer
senkrechter Strich in den Balkendiagrammen zu finden. Die gestrichelten Linien stellen das Mittel über die Mittelwerte
aller  evaluierten Vorlesungen bei der jeweiligen Frage dar (s. Beispiel). Dabei wurden alle Vorlesungen gleich gewichtet,
unabhängig von der Teilnehmerzahl. Der Vergleich zwischen dem Mittelwert einer Vorlesung und dem Gesamtmittel ist in vielen
Fällen aussagekräftiger als der Absolutwert des Mittelwertes. Es sollte aber beachtet werden, daß sich manche Vorlesungen
aufgrund völlig unterschiedlicher Situationen nicht vergleichen lassen. (Z.B. geben die Versuche in Physik I der Vorlesung
unabhängig vom Vortragsstil einen Bonus.)

\vspace{3mm}
\textbf{Beispiel:}
\vskip8mm\centerline{\parbox{80mm}{\epsfxsize=80mm\epsffile{vorwort.eps}}}\vskip6mm
\vspace{5mm}

\textbf{\large Zu den Kommentaren:}
\vspace{0.05cm}

Die Kommentare der Studierenden sind im Anschluß an die Fragen aufgeführt. Persönlich beleidigende Anmerkungen und Äußerungen, die
nichts mit der Vorlesung zu tun haben, wurden weggelassen. In einigen Fällen haben wir Kommentare nur zusammengefasst und in Form einer
\emph{Anm. d. Fachschaft} wiedergegeben. Teilweise wurden zur besseren Lesbarkeit grammatikalische und orthographische Dienste
geleistet. In ihrer Bedeutung gleiche Bemerkungen wurden zusammengefaßt und  die Häufigkeit ihrer Nennung in Klammern dahintergesetzt.

\vspace{5mm}
\textbf{\large Allgemeine Bemerkungen zur Berurteilung der Lehrveranstaltungen:}
\vspace{0,05cm}

Wie man sieht, liegen die Mittelwerte über alle Vorlesungen meist höher als die neutralen Werte. Die Veranstaltungen scheinen von den
Studierenden geringfügig besser als eine durchschnittliche Vorlesung empfunden zu werden. Wie vorhin schon erwähnt muß noch der Anteil
der Studierenden, die den Besuch der Veranstaltungen im Laufe dieses Semesters aufgegeben haben und deshalb nicht von der Umfrage erfaßt
worden sind, beachtet werden.

\vspace{5mm}

\textbf{\large Und was passiert jetzt damit?}
\vspace{0.05cm}

Die Ergebnisse der Umfrage wurden (wie Ihr seht) veröffentlicht. Außerdem bekommt jeder Professor eine Auswertung von uns persönlich
überreicht, mit der Bitte, sie in der Vorlesung zu besprechen --- daran solltet Ihr ihn gelegentlich erinnern. Die Erfahrung zeigt,
dass konstruktive Kritik (wie sie ja auch vielfach in den Kommentaren geübt wurde)
eigentlich immer gut ankommt. Sollten sich trotzdem irgendwelche Probleme ergeben, könnt ihr Euch natürlich auch an die Fachschaft wenden.

\vspace{10mm}

{\sl Marburg am \today
  \\
  Die Fachschaft Physik der Philipps-Universität Marburg}

\newpage

% TODO: maschinell generieren
\BLOCK{for vl in vorlesungen}

\Kopfgross{\VAR{vl.name}}{\VAR{ vl.dozenten.all()|concat}}{\VAR{vl.antwortboegen.all()|count}}
\BLOCK{if vl.antwortboegen.all()|count > 0}

\begin{center}
  \begin{tabular}{p{3cm}p{3cm}p{5cm}l}
    {\bf Semesterverteilung:}&&{\bf Studiengänge:}& \\
    \BLOCK{for index in range(vl.studiengang_dist()|max(vl.semester_dist()))}
    \BLOCK{if vl.semester_dist()|count > index -}
      Semester \VAR{vl.semester_dist()[index]['key']}: & \VAR{vl.semester_dist()[index]['val']} 
    \BLOCK{- else} & \BLOCK{endif -}
    &
    \BLOCK{- if vl.studiengang_dist()|count > index -}
      \VAR{vl.studiengang_dist()[index]['key']}: & \VAR{vl.studiengang_dist()[index]['val']} 
    \BLOCK{- else} & \BLOCK{endif} \\
    \BLOCK{endfor}
  \end{tabular}
\end{center}
\Strich

% auswertung
\BLOCK{for fragenset in vl.fragebogen.fragensets.all()}
  \BLOCK{for frage in fragenset.fragenn(fragen)} 
    \BLOCK{if not frage.fragentyp.texttype and frage.total(vl) > 0}
      \Frage{\VAR{frage.text}}{\VAR{frage.total(vl)}}
      \BLOCK{- for option in frage.fragentyp.fast_optionen(optionen) -}
        \BLOCK{- if loop.first -}
          {\VAR{option.text|latexify}}{\Stats 
        \BLOCK{- endif -}
        {\VAR{(option.value(vl, frage) * 100)|int}}
        \BLOCK{- if loop.last -}
          {\VAR{((frage.mean(vlu)-1) * 100)|int}}
          {\VAR{((frage.mean(vl)-1) * 100)|int}}}
          {\VAR{option.text|latexify}}
        \BLOCK{- endif -}
      \BLOCK{- endfor} \\
    \BLOCK{- endif}
  \BLOCK{endfor}
\BLOCK{endfor}

\Strich
\centerline{\large Liste aller Kommentare:}

\BLOCK{for fragenset in vl.fragebogen.fragensets.all() -}
  \BLOCK{- for frage in fragenset.fragenn(fragen) -}
    \BLOCK{- if frage.fragentyp.texttype and frage.id != 34}
      {\bf \VAR{frage.text}}
      \BLOCK{if frage.antworten(vl)|count > 0 -}
        \begin{itemize}
          \setlength{\itemsep}{0cm} \setlength{\parsep}{0cm}
        \BLOCK{for k in frage.antworten(vl)}
          \item \VAR{k.text|latexify}
        \BLOCK{endfor}
        \end{itemize}
      \BLOCK{else}
        \begin{itemize}\item Keine Kommentare\end{itemize}
      \BLOCK{endif -}
    \BLOCK{- endif -}

    \BLOCK{- if frage.fragentyp.texttype and frage.id == 34}
      {\bf \VAR{frage.text}}
      \BLOCK{if vl.tutoren.all()|count > 0}
        \begin{itemize}
        \BLOCK{for tutor in vl.tutoren.all()}
          \item Gruppe von \VAR{tutor.name}
          \BLOCK{if frage.antworten(vl, tutor)|count > 0 -}
            \begin{itemize}
              \setlength{\itemsep}{0cm} \setlength{\parsep}{0cm}
            \BLOCK{for k in frage.antworten(vl, tutor)}
              \item \VAR{k.text|latexify}
            \BLOCK{endfor}
            \end{itemize}
          \BLOCK{- else -}
            \begin{itemize}\item Keine Kommentare\end{itemize}
          \BLOCK{endif -}
        \BLOCK{- endfor -}
        \end{itemize}
      \BLOCK{else}
        \begin{itemize}\item Keine Kommentare\end{itemize}
      \BLOCK{- endif}
    \BLOCK{endif}
  \BLOCK{- endfor -}
\BLOCK{- endfor}

\BLOCK{endif}

\newpage

\BLOCK{endfor}

\end{document}

